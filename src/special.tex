\subsection{Complete Graphs}
Consider the complete graph $K_n$. In this case, $D = \frac{1}{n-1}I$ and
$A = (J - I)$, so $T=\frac{1}{n-1}(J-I)$. In order to compute $T^ke_u$ in the
limit, we will represent $e_u$ as a linear combination of eigenvectors of $A$.

\begin{remark}
  $1_n$ is an eigenvector of $T$ with eigenvalue $\lambda = 1$, and every
  vector $x \in \R^n$ such that $\sum_i x_i = 0$ is an eigenvector of $T$ with
  $\lambda = -\frac{1}{n-1}$.
\end{remark}
\begin{proof}
  Multiplying a vector by the all-ones matrix takes the sum of that vector's
  values for every index in the result, so $J \cdot 1_n = n \cdot 1_n$.
  Similarly $Jx = 0_n$, for any $x$ s.t. $\sum_i x_i = 0$. Thus,

  \begin{align*}
    T\cdot 1_n &= \frac{1}{n-1}(J-I) \cdot 1_n \\
               &= \frac{1}{n-1}(n\cdot 1_n - 1_n) \\
               &= 1_n
  \end{align*}

  and

  \begin{align*}
    Tx &= \frac{1}{n-1}(J-I)x \\
       &= \frac{1}{n-1}(0_n - x) \\
       &= -\frac{1}{n-1}x
  \end{align*}
\end{proof}


Next, we will show how to write any $e_u$ as a linear combination of
eigenvectors of $T$. We define $\alpha_u$ by $\alpha_{u,u} := n-1$ and
$\alpha_{u,j} = -1, u\neq j$. The entries of $\alpha_u$ sum to $0$, so
$T\alpha_u=-\frac{1}{n-1}\alpha_u$. We can write
$e_u = \frac{1}{n}(1_n + \alpha_u)$.

\begin{corollary}
  The eigenvectors of $J-I$ span $\R^n$.
\end{corollary}
\begin{proof}
  Since all standard basis elements $e_j$ of $\R^n$ can be expressed as linear
  combinations of eigenvectors for $J-I$, the eigenvectors of $J-I$ span $\R^n$.
\end{proof}

\begin{theorem}
  Let $K_n$ be the complete graph with nodes labelled $1,...,n$. Then for any
  two distinct nodes $u$ and $v$, $\delta(u,v) = \frac{2(n-1)}{n}$.
\end{theorem}
\begin{proof}
  We will use $\alpha_u$ as defined above. DSD is given by

\begin{align*}
  \delta(u,v) &= \sum_{k = 0}^{\infty}{||T^ke_u - T^ke_v||_1} \\
              &= \frac{1}{n}\sum_{k = 0}^{\infty}{||T^k(1_n + \alpha_u - 1_n -
                \alpha_v)||_1}\\
              &= \frac{1}{n}\sum_{k = 0}^{\infty}{||T^k(\alpha_u - \alpha_v)||_1} \\
              &= \frac{1}{n}\sum_{k = 0}^{\infty}{||(-\frac{1}{n-1})^k(\alpha_u -
                \alpha_v)||_1} \\
              &= \frac{||\alpha_u - \alpha_v||_1}{n}
                \sum_{k = 0}^{\infty}{(-\frac{1}{n-1})^k} \\
\end{align*}

Since $|-\frac{1}{n-1}| < 1$, the geometric series converges,
$\sum_{k=0}^{\infty}(-\frac{1}{n-1})^k = \frac{n-1}{n}$. When $u \neq v$, the
difference $\alpha_u - \alpha_v$ will have exactly two non-zero entries (since
they are identical at all indices but $u$ and $v$). These non-zero entries will
be $n$ at index $u$ and $-n$ at index $v$. Thus,
$||\alpha(u)-\alpha(v)||_1 = 2n$, and so

\begin{align*}
  \delta(u,v) &= \frac{2n}{n}(\frac{n-1}{n}) \\
              &= \frac{2(n-1)}{n} \\
\end{align*}

\end{proof}

\subsection{Circulant Graphs}

Circulant graphs have well-understood spectra, so we can perform a similar
analysis to the complete case.

\begin{definition}
  We {\bf rotate} a row vector by moving each of its elements over one index to
  the right. The rightmost element wraps around and is put in the leftmost index
  in the result.
\end{definition}

\begin{example}
  The rotation of $\colvec{1 & 2 & 3 & 4}$ is $\colvec{4 & 1 & 2 & 3}$.
\end{example}

\begin{definition}
  A matrix is {\bf circulant} if each row is equal to the A graph is {\bf
    circulant} if it has a circulant adjacency matrix.
\end{definition}

\begin{example}
  The pentagon with this circulant adjacency matrix:
  \[
    \begin{bmatrix}
      0 & 1 & 0 & 0 & 1 \\
      1 & 0 & 1 & 0 & 0 \\
      0 & 1 & 0 & 1 & 0 \\
      0 & 0 & 1 & 0 & 1 \\
      1 & 0 & 0 & 1 & 0 \\
    \end{bmatrix}
  \]
\end{example}

\subsubsection*{Spectrum}

[The expressions for eigenvectors/eigenvalues are sourced from Wikipedia --JP]

All circulant matrices have the same eigenvectors given by

\[x_j = \colvec{1 \\ z^j \\ z^{2j} \\ ... \\ z^{(n-1)j}}\]

where $j=1,2,...,n$ and $z=\exp(\frac{2 \pi i}{n})$ is an $n^{th}$ root of
unity. Note that the $n^{th}$ eigenvector is $1_n$, and that multiplying $x_j$
by $z^{n-j}$ rotates it, as $1 = z^0 = z^n$. We can write our basis vectors $e_u$ as
linear combinations of elements in $x_j$ like so:

\[e_u = \frac{1}{n}(1_n + \sum^{n-1}_{j=1} z^{(u-1)(n-j)} x_j)\]

This depends on the fact that the sum of all $n^{th}$ roots of unity is 0, so
$1 + \sum_{j=1}^{n-1}z^j = 0$. Therefore, we get $ne_1$ by summing over all
eigenvectors, since every index except the first will vanish. We can get the
other basis vectors by repeatedly rotating all the eigenvectors except $1_n$,
which is where the $z^{(u-1)(n-k)}$ comes from in the form above.

The corresponding eigenvalues $\lambda_j$ depend on the matrix. Let
$r = \colvec{r_1 & r_2 & ... & r_n}$ be the first row of the matrix. Then, the
$j^{th}$ eigenvalue is given by

\[\lambda_j = \sum_{k}r_kz^{(k-1)j} \]


\subsubsection*{Cycles}

Cycles are the most obvious example of circulant graphs. The eigenvalues of a
cycle are given by $\lambda_j = z^{j-1} + z^{j+1}$.
