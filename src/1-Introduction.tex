An abundance of network data has led to many studies on predicting information about vertices in networks. Such studies are important because they can lead to recommendation systems ~\cite{huang2004graph}, classification of vertices in the network, the prediction of protein function, and many other useful applications. Many graph-based approaches have been considered for prediction on networks, as networks have a natural relation to graphs. In this project, we focus on a specific method of prediction, previously used in a biological network, and try to prove why it is more useful for prediction in certain networks than other methods of prediction.

Suppose we have a graph in which every vertex belongs to a unique class and a map from a proper
subset of vertices to labels corresponding to their classes. Given such a graph, we define the vertex label prediction problem as
predicting the missing labels in this graph. This problem has great practical importance. Data with a network
structure is ubiquitous in real problems involving social media, the world wide web, and the biology
of genes and proteins, and label prediction presents interesting possibilities ~\cite{FRASCA201384}.

One such possibility is presented by protein protein-interaction (PPI) networks. A PPI network is a
graph where each vertex represents a protein and edges are drawn between any two proteins which are
known to interact. In this case, some proteins are labeled based upon their function in the
organism, and the task is to predict functions for the unlabeled proteins based upon the PPI
network.
%% TODO: Summarize history & current state of the art? (i.e. competing with Cao et al)
To tackle this problem, Cao, Zhang, Park, Daniels, Crovella, Cowen, and Hescott developed a new
metric on graph vertices, the diffusion state distance (DSD). They used the DSD to build a predictor
that provided better performance of protein function predictors than competing methods based upon shortest path distance
~\cite{10.1371/journal.pone.0076339}. %which competing methods?!

% need here: a description of the problem: (analysis of DSD is limited)

% ?TODO: incorporate here numeric information about PPI data set used by Cao et al.

The goal of our project was to further explore graph-metric-based classification approaches in order to
better understand the aspects of graph structure to which DSD reacts in comparison to other graph metrics. To accomplish this, we
generalize the weighted nearest neighbors prediction method used by Cao et al. to use arbitrary
graph metrics, and then performed a number of classification experiments using both simulated and
real-world data sets.

We succeeded in developing novel generative models for labeled graphs which have a recoverable
community structure. In addition, we show that DSD-based classification is more robust to noise and
performs better all around than the other metrics do in a variety of different classification
experiments. We conclude that the DSD is a robust and meaningful measurement of distance on graphs
with certain sorts of underlying community structure.

Chapter 2 provides the background in graph theory needed to understand our methods, while Chapter 3
presents the novel data simulation models developed to test vertex label prediction algorithms. In
Chapter 4, we develop background in metrics on graphs, particularly the DSD. These metrics are the
keystone of our weighted nearest neighbors label prediction approach, which is described in Chapter
5 alongside a discussion of related machine learning methods of graphs.

Chapter 6 details the methods, results, and analyses of a suite of classification experiments
performed on data simulated using the models described in Chapter 3. Chapter 7 builds upon these
results with further experiments run on two real data sets.


%%% Local Variables:
%%% mode: latex
%%% TeX-master: "../Main.tex"
%%% End:
