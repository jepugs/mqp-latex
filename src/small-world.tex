\section{Elementary Graph Theory}

We begin by recalling some conventional definitions and notations from graph
theory.

\begin{definition}
  A \textbf{graph} $G$ is a pair of sets $(V,E)$ such that $E = V \times V$,
  where equality on $E$ disregards order, that is $(u,v) = (v,u)$. We call $V$
  the \textbf{vertices} of $G$ and $E$ the \textbf{edges} of $G$.
\end{definition}

For our purpose, $V$ will always be a sequence of natural numbers starting from
$1$, that is

\[
  V = \{ n \in \mathbb{N} : 1 \leq i \leq N \}
\]

for some $N$.

\begin{remark}[Notation]
  Let $G$ be a graph.

  \begin{itemize}
  \item We access the vertices and edges of G by the functions $V(G)$ and $E(G)$
    respectively.
  \item $v \in G$ is an equivalent statement to $v \in V(G)$
  \item $|G|$, called the \textbf{order} of $G$, is equal to $|V(G)|$.
  \end{itemize}
\end{remark}

\begin{definition}
  The \textbf{degree} of a vertex $u \in G$, denoted $\deg (u)$, is the number
  of edges in $G$ which contain $u$.
\end{definition}

\begin{definition}
  The \textbf{adjacency matrix} $A$ of a graph $G$ is given by
  \[
    A_{ij} = \begin{cases}
      1 &: (i,j) ~\text{is an edge in $G$} \\
      0 &: \text{otherwise} \\
    \end{cases}
  \]
\end{definition}

Note that the adjacency matrix of a graph is not unique in general. Depending on
the way that the vertices of $G$ are numbered, we may end up with a different
matrix.

\begin{definition}
  The \textbf{degree matrix} of $G$ is the diagonal matrix defined by
  \[
    D_{ij} = \begin{cases}
      \deg(i) &: i = j \\
      0 &: i \neq j
    \end{cases}
  \]
\end{definition}

\begin{definition}
  The \textbf{graph laplacian} of $G$ is given by $L = D - A$, where $D$ is the
  degree matrix of $G$ and $A$ is the adjacency matrix. In other words,

  \[
    L_{ij} = \begin{cases}
      \deg(i) &: i=j \\
      -1 &: (i,j) ~\text{is an edge in $G$} \\
      0 &: \text{otherwise}
    \end{cases}
  \]
\end{definition}

\section{Small-World and Scale-free Networks}

The motivation for DSD comes from small-world networks, which are a class of
graphs characterized by having a very small average shortest-path distance
between any two vertices compared to the overall size of the graph.


\begin{definition}
  The \textbf{characteristic path length} of a graph $G = (V,E)$ is the average
  distance between vertices in the graph, which is given by

  \[ L_G = \frac{1}{|E|} \sum_{\substack{u,v \in V \\ u \neq v}} \dist(u,v)\]

  where $\dist(u,v)$ is shortest-path distance. 
\end{definition}

\begin{definition}
  Consider a graph $G = (V,E)$. $G$ is a \textbf{small-world network} if
  $L_G \sim \log{|V|}$.
\end{definition}

Small-world networks are ubiquitous in a variety of applications areas. However,
many real-world examples of such graphs also have other distinctive structural
properties which are not necessarily captured by their small-worldedness alone.

% We can partially describe this structure in terms of the
% clustering coefficient.

% \begin{definition}
%   Let $v$ be a vertex in a graph $G = (V,E)$.

%   Consider the subgraph formed by the neighborhood of $v$. This subgraph has at
%   most $\deg(v)(\deg(v) - 1)$ edges, which happens when it is a clique. Denote
%   the number of edges of the neighborhood subgraph by $E_{N(v)}$.

%   The \textbf{clustering coefficient} $C_v$ is defined by
%   $C_v = \frac{E_{N(v)}}{\deg(v)(\deg(v) - 1)}$, that is, $C_v$ is the
%   proportion of edges in the neighborhood subgraph out of all possible edges.
%   The clustering coefficient of $G$ is the average of all vertex clustering
%   coefficients, $C_G = \frac{\sum_{v \in V}{C_v}}{|V|}$.
% \end{definition}

\begin{definition}
  A \textbf{scale-free network} is a graph whose vertex degrees follow a power
  law. That is, given a randomly selected $u \in G$, $P(\deg(u) = k) \sim
  k^{-\gamma}$ for some $\gamma > 0$.
\end{definition}

One of the defining characteristics of power law distributions is that they have
very long tails. In a scale-free graph with many vertices, this implies the
existence of \textbf{hubs}, which we informally define as vertices whose degrees
are far larger than average in the graph.

\begin{theorem}[R. Cohen and S. Havlin]
  Scale-free networks are small-world networks.
\end{theorem}

Real world examples of scale-free networks are abundant and include the world
wide web, cellular communication networks, and protein-protein interaction (PPI)
networks. It is not surprising, then, that scale-free networks are necessarily
small-worlds networks, however we defer to Cohen and Havlin for proof of this
fact ~\cite{PhysRevLett.90.058701}.


\section{Random Graphs}

\begin{definition}
  A \textbf{random graph} is a random variable for which all outcomes are
  undirected graphs.

  A \textbf{random graph process}, denoted $(G_t)$, is a family of random graphs
  indexed by a discrete time $t \in \mathbb{N}$.
\end{definition}

We are interested in random graph processes which build small-worlds networks.
The Watts-Strogatz graphs are the earliest example of such a process. They are
defined by starting with a regular ring lattice and randomly rewiring edges
until the graph is obtained.

\begin{definition}
  An $n$-$k$ \textbf{regular ring lattice} is a graph $(V, E)$ with $n$ vertices
  such that $u,v$ is an edge if and only if $|u-v| \leq \frac{k}{2}$.
\end{definition}

\begin{definition}
  Let $v$ be a vertex in a graph. We can \textbf{rewire} $v$ by deleting one
  edge of $v$ and replacing it with 
\end{definition}

\begin{definition}
  Given a probability $p$ and regular ring lattice parameters $n$ and $k$, we
  define the $(n,k,p)$ \textbf{Watts-Strogatz Process} as follows.

  Let $r(G,v)$ be a random process that rewires vertex $v$ in a graph $G$. We
  define a family of graphs, $(G_t)$, by

  \[
    G_{t+1} = \left\{
      \begin{array}{lc}
        r(G_t,t) &: X \leq p \\
        G_t &: X > p
      \end{array}
    \right.
  \]

  for $t = 1,\dots, |G|$, where $G_0$ is the $n$-$k$ regular ring lattice, and
  $X$ is a uniform random variable in the range $[0,1]$. That is, we iterate
  over the vertices of the ring lattice and rewire each one with probability
  $p$.

  We call graphs sampled from $G_{|G|}$ \textbf{Watts-Strogatz graphs}.
\end{definition}

The Watts-Strogatz graphs are small-world networks for all but extremely small
values of $p$ ~\cite{Watts1998Collective}. However, in general, they are not
scale free networks ~\cite{Barabasi509}, and thus do not show the structural
characteristics of many practical data sets.

\begin{definition}
  Let any graph $G_0$ be given as well as some parameter $m$, $m \leq |G_0|$. We
  build the random graph $G_{t+1}$ by adding a new vertex to $G_t$ and
  connecting it to $m$ vertices of $G_t$ with probabilities proportional to the
  degree of each vertex. That is, the probability of adding an edge to a vertex
  $u$ is

  \[
    p_u = \frac{\deg(v)}{\sum_{v \in G_t} \deg(v)}
  \]

  on the first step, and this is done a total of m times without replacement.

  We call $(G_t)$ the $(G_o,m)$-\textbf{Barab\'asi-Albert (BA) process} and graphs
  sampled from $G_n$ $(G_o,m,n)$-\textbf{Barab\'asi-Albert (BA) graphs}.
\end{definition}

This type of model, where edges to a new vertex are drawn with non-uniform
probability, is known as a \textbf{preferential attachment} model.

\begin{theorem}[A. Barab\'asi and R. Albert]
  \textbf{Barab\'asi-Albert graphs} are scale-free
\end{theorem}

In the context of classification problems, one issue that arises is that these
generative models do not suggest any natural labeling for the vertices of these
graphs. Thus, we propose a variant on the BA model which yields a labeled graph.

\begin{definition}
  Let parameters $G_0$ and $m \leq |G_0|$ be given as in the BA process as well
  as a finite set of labels $S$ and a factor $\rho \ge 1$. In addition, suppose
  each vertex in $G_0$ has been labeled by an element of $S$. Given a graph
  $G_t$, we build $G_{t+1}$ as follows.

  As in the BA process, we will add one new vertex and draw edges using
  preferential attachment, however, we reweight the probability of drawing each
  edge in order to favor vertices with label $l$ by a factor of $\rho$. We label
  our new vertex $l$, which is chosen from $S$ with uniform probability.

  Define $V_l$ to be the set of vertices with label $l$. Then, we compute the
  probability of drawing an edge to a vertex $u$ as

  \[
    p_u = \begin{cases}
      \frac{\rho\deg(u)}{w} &: u \in V_l\\
      
      \frac{\deg(u)}{w} &: u \in (V(G) - V_l)\\
    \end{cases}
  \]

  where $w$ is the weighted degree sum,

  \[
    w = \rho \sum_{v \in V_l}\deg(v) + \sum_{v \in (V(G) - V_l)}\deg{v}
  \]

  We call this random process the \textbf{class-weighted Barab\'asi-Albert
    (CWBA) process} and we call graphs sampled from this model \textbf{CWBA
    graphs}.
\end{definition}


%%% Local Variables:
%%% mode: latex
%%% TeX-master: "../Main"
%%% End:
