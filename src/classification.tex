In this chapter, we present how we can apply the classification problem to labeled random graphs using metrics on graphs such as the Diffusion State Distance (DSD). We present our problem formulation and its relation to the topics discussed in the previous chapters.

\section{Problem Definition}
Consider an undirected connected graph $G = (V,E)$ (Chapter 2 Definition 1). A graph is \textbf{undirected} if for all edges $(u,v) \in E$, where $u,v \in V$ we have $(u,v) = (v,u)$. A graph is \textbf{connected} if there is a path (Chapter 2 Definition 4) from any $u \in V$ to any other $v \in V$, $u \neq v$. Many real world examples exist of data that can be represented as undirected graphs, and will be discussed later in this chapter. Such data may also contain information about attributes of the vertices in the graph. Yang and Leskovec~\cite{DBLP:journals/corr/abs-1205-6233} used the scientific collaboration network DBLP where authors were modeled as vertices and edges were formed between authors who had published at least one paper together. Vertices were labeled by publication venues. A construction of a graph such as this gives us a labeled graph. However, not all labels of vertices may be known. Cao, Zhang, Park, Daniels, Crovella, Cowen, and Hescott~\cite{10.1371/journal.pone.0076339} study the prediction of protein function on protein-protein interaction networks. Many protein functions, labels on the vertices of these networks, are not known. Thus, prediction algorithms are used to predict protein functions on these networks. A variety of prediction algorithms exist, but we study prediction algorithms that incorporate metrics on graphs. Lastly, prediction accuracies are computed for prediction algorithms using different metrics.
