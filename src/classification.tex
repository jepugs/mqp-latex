In this chapter, we define label prediction methods and show how they can be applied on graphs. We present different problems we can solve using label prediction methods and how we can incorporate graph metrics, such as the Diffusion State Distance (DSD), into these methods. We use the terms label prediction methods and vertex classifiers interchangeably.

% severely fix
\section{Problem Definition}
\label{sec:label_prediction_methods}
We begin by presenting definitions relating to labels of graphs.

\begin{definition}
A \textbf{partially labeled graph} is a graph $G$ and a function $c : V_{G_p} \to S$,
where $S$ is an arbitrary set of labels, and $V_{G_p} \subseteq V_G$ with
$|V_{G_p}| = \left\lfloor{p \cdot |V_G|}\right\rfloor$, $0 \leq p \leq 1$.

The set $V_G \setminus V_{G_p}$ is called the set of \textbf{unlabeled} vertices of $G$.

The function $c$ is called a \textbf{partial labeling} of $G$.
\end{definition}

\begin{definition}
A \textbf{censoring} of a labeled graph $G$ and its labeling function $c : V_G \to S$
is a set $V_{G_c}$ and a function $c_p: V_G \setminus V_{G_c} \to S$ such that
$c_p$ is a partial labeling of $G$.

The function $c$ of the labeled graph $G$ is called the \textbf{ground truth labeling} of the censoring of $G$.

Note that the set $V_{G_c}$ is the set of unlabeled vertices of the partial labeling of $G$.
\end{definition}

Given a partially labeled graph $G$, a censoring $(V_{G_c}, c_{censor})$ of $G$,
and a ground truth labeling $c_{ground truth}$ of $G$, a label prediction method is
an algorithm that attempts to correctly guess the labels of the unlabeled vertices $V_{G_c}$
with respect to the ground truth labels $c_{ground truth}(v), v \in V_{G_c}$.

We now discuss several label prediction methods that we considered for our simulations in Chapter 6.

% Fix wording make more mathematical
\subsection{Majority Voting Algorithm}
Cao, Zhang, Park, Daniels, Crovella, Cowen, and Hescott~\cite{10.1371/journal.pone.0076339}
mention a simple prediction method called the neighborhood majority voting algorithm.
We considered two implementations of this algorithm. One implementation considers
each vertex $v \in V$ and all neighbors of $v$ within an $\varepsilon$ distance
from $v$ (a ball of radius $\varepsilon$). The $\varepsilon$ distance depends on the metric under consideration, and may be changed as a parameter. In an unweighted scheme, each neighbor within the ball of radius $\varepsilon$ votes equally for their own label. In a weighted scheme, each neighbor gets a vote proportional to the reciprocal of their distance to the vertex $v$ in consideration. The other implementation of this algorithm considers each vertex $v \epsilon V$ and the $k$-nearest neighbors of $v$. Voting is done similarly in both an unweighted and weighted scheme.




 Many real world examples exist of data that can be represented as undirected graphs, and will be discussed later in this chapter. Such data may also contain information about attributes of the vertices in the graph. Yang and Leskovec~\cite{DBLP:journals/corr/abs-1205-6233} used the scientific collaboration network DBLP where authors were modeled as vertices and edges were formed between authors who had published at least one paper together. Vertices were labeled by publication venues. A construction of a graph such as this gives us a labeled graph. However, not all labels of vertices may be known. Cao, Zhang, Park, Daniels, Crovella, Cowen, and Hescott~\cite{10.1371/journal.pone.0076339} study the prediction of protein function on protein-protein interaction networks. Many protein functions, labels on the vertices of these networks, are not known. Thus, prediction algorithms are used to predict protein functions on these networks. A variety of prediction algorithms exist, but we study prediction algorithms that incorporate metrics on graphs. Lastly, prediction accuracies are computed for prediction algorithms using different metrics.
