Since we wish to use random graphs to test vertex classifiers, an issue that
arises is that the generative models presented so far do not suggest any natural
labeling for the vertices of the resultant graphs. Thus, we propose some labeled
models for generating small-world and scale-free networks.

\begin{definition}
  Let $m$ be a natural number. We construct the \textbf{complete components
    graph} of order $n = 2m$ as follows. Let $V = \{1,2, ..., 2m\}$, and then define
  $E$ by

  \[
    E = \{ (u,v) : u,v \in \{1,...,m\} ~\text{or}~ u,v \in \{m+1,...,2m\} \}
  \]

  i.e. the disjoint union of two complete graphs of order $m$.
\end{definition}

\begin{definition}
  Let parameters $(m,p,q)$ be given, $m \in \mathbb{N}$ and $p,q \in [0,1]$, and
  let $G_0$ be the complete components graph of order $2m$. We construct a
  \textbf{Noisy Complete Components (NCC)} graphs as follows.

  Iterate over ever pair of vertices in the graph (i.e. the edge set of
  $K_{2m}$). For every pair $(u,v) \in E_{G_0}$, that is, for the edges that
  already exist in $G_0$, we delete the edge with probability $q$. Likewise, for
  each pair $(u,v) \notin E_{G_0}$, we add the edge $(u,v)$ with probability
  $p$.

  The resultant graph is an NCC graph.
\end{definition}

For most values of $p$ and $q$, the resulting NCC graphs will be small world
networks. However, these graphs are extremely regular, so they do not exhibit
scale-free properties. Thus, we also propose a variant on the Barab\'asi-Albert
model which adds labels while preserving the power law degree distribution.

%% prove theorem: the following are equal: frobenius norm of a matrix,
%% \sum{\lambda^2}, number of edges in G (or something like this)

% sketch: recall \sum \lambda_i = \tr(M). Consider A^2, whose trace is the sum
% of degrees

\begin{definition}
  Let parameters $G_0$ and $m \leq |G_0|$ be given as in the BA process as well
  as a finite set of labels $S$ and a factor $\rho \ge 1$. In addition, suppose
  each vertex in $G_0$ has been labeled by an element of $S$. Given a graph
  $G_t$, we build $G_{t+1}$ as follows.

  As in the BA process, we will add one new vertex and draw edges using
  preferential attachment, however, we reweight the probability of drawing each
  edge in order to favor vertices with label $l$ by a factor of $\rho$. We label
  our new vertex $l$, which is chosen from $S$ with uniform probability.

  Define $S_l$ to be the set of vertices with label $l$. Then, we compute the
  probability of drawing an edge to a vertex $u$ as

  \[
    p_u = \begin{cases}
      \frac{\rho\deg(u)}{w} &: u \in S_l\\
      
      \frac{\deg(u)}{w} &: u \in (V_G - S_l)\\
    \end{cases}
  \]

  where $w$ is the weighted degree sum,

  \[
    w = \rho \sum_{v \in S_l}\deg(v) + \sum_{v \in (V_G - S_l)}\deg(v)
  \]

  We call this random process the \textbf{class-weighted Barab\'asi-Albert
    (CWBA) process} and we call graphs sampled from this model \textbf{CWBA
    graphs}.
\end{definition}

%%% Local Variables:
%%% mode: latex
%%% TeX-master: "../Main"
%%% End: