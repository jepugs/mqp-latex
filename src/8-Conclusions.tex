Our project hopes to provide information about what makes prediction methods using the diffusion
state distance (DSD) metric work better than methods using other metrics on graphs. Our simulation
models were constructed to attempt to identify structures within graphs that caused the DSD to work
better. In Chapter 3, we proposed the novel Noisy Complete Components (NCC) and Noisy Complete
Components with Hubs (NCCH) models in order to see whether tightly clustered graphs and dense
neighborhoods would affect prediction using the DSD. We also developed the Class-Weighted
Barab\'{a}si-Albert (CWBA) model in order to see the effects of scale-free properties on prediction
using the DSD.

Using our simulation models, we experimented with various parameters of our models in order to get data about changes in prediction accuracy with respect to changes in the graph structure (Chapter 6). We tested parameters that affected neighborhoods of vertices in our models as well as the number of hub vertices. More models could be constructed in future work to draw out characteristics of graph structure that the DSD metric is most affected by. Also, subject-knowledge for specific types of networks, mentioned in Chapter 5, could be added to the models to add to our metric-based clustering algorithm.

Our simulation results show that the DSD is more effective at detecting some 
properties of graph structure than the shortest path distance and the
resistance distance for dense graphs. Prediction methods using the DSD were 
shown to be more robust to hub vertices and our NCC simulations were able to
intuitively show how the DSD determines communities based on neighborhoods 
rather than direct neighbors. Our project could have included simulations to
study the resistance distance metric as well, and graph models that would
cause the DSD metric to cause prediction accuracies to be significantly 
worse than the shortest path distance.

Our analysis of real-world networks (coauthor, email-Eu-core) did not provide
as much information as we expected on how the DSD metric is useful on examples
of real networks. However, both still showed that the DSD metric performed
best of the three metrics that we studied.

There are many interesting questions remaining. For what parameters for CWBA graphs would the SPD perform better than the DSD? We believe that such 
an investigation would provide further insight into properties of scale-free networks which would indicate the most appropriate metric for classification
problems on scale-free networks. This analysis could also examine other 
statistics on graphs such as betweenness centrality~\cite{newman2005measure} and eigencentrality based on dissimilarity measures~\cite{alvarez2015eigencentrality}
to see whether they are correlated with DSD performance. The effectiveness of the DSD in detecting manifold-like structures of
graphs is also a challenge that could be explored. Several simulation models could be
constructed to study this question. Also, the effect of the DSD metric in
detecting the dispersion of rumors or the spread of information could also
be studied. This would relate to the virality of news, products, clothing, and other trends. The way the DSD metric is defined by random walks seems to give
a hint for what kinds of real-world data that it should be experimented with.
The DSD metric could be compared to heat dispersion and different kinds
of dispersions in nature, such as the dispersion of particles or biological
dispersion of pollen and seeds. Such studies could provide a more accurate
understanding of how natural biological networks interact with each other.