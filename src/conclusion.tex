Our simulation results show that the DSD is more effective at detecting some 
properties of graph structure than the shortest path distance and the
resistance distance for dense graphs. Prediction methods using the DSD were 
shown to be more robust to hub vertices and changes in graph structure, such
as addition and removal of edges in a graph. The DSD was shown to be able to
capture communities based on neighborhoods rather than direct neighbors.

Our eu-Email-core network results show that the DSD is more effective at 
detecting community structure in real-world networks. Our coauthor network 
results show this as well, but this may be due to the spectral clustering
algorithm SCORE, and may say more about how SCORE does not generate 
realistic communities.

For future work we suggest creating more simulation models to find more
specific properties of graph structure that cause the DSD to be more
effective at detecting communities. Also more simulations on real-world data
could be performed to further test the effectiveness of the DSD on real-world
data.

Overall, prediction methods using the DSD can be used for social networks,
biological networks, or any network with an underlying graph community
structure.