Our simulations models were constructed to attempt to bring out the parts of 
structures of graphs that would make prediction methods using the DSD metric 
work better than other metrics on graphs. We came up with our NCC, NCCH, and
CWBA models in order to see the effect of dense neighborhoods and a scale-
free structure of graphs on prediction using the DSD. We varied parameters of 
our models in order to get data about changes in prediction accuracy with
respect to changes in the graph structure. More models could be constructed
in future work to draw out characteristics of graph structure that the DSD
metric is most affected by. Also, more data, such as subject-knowledge for
specific types of networks, could be added to the models to add to our
metric-based clustering algorithm.

Our simulation results show that the DSD is more effective at detecting some 
properties of graph structure than the shortest path distance and the
resistance distance for dense graphs. Prediction methods using the DSD were 
shown to be more robust to hub vertices and our NCC simulations were able to
intuitively show how the DSD determines communities based on neighborhoods 
rather than direct neighbors. Our project could have included simulations to
study the resistance distance metric as well, and graph models that would
cause the DSD metric to cause prediction accuracies to be significantly 
worse than the shortest path distance.

Our analysis of real-world networks (coauthor, email-Eu-core) did not provide
as much information as we wanted on how the DSD metric is useful on examples
of real networks. However, both still showed that the DSD metric performed
best of the three metrics that we studied.

% Add citations here?
There are many interesting questions that could still be answered regarding 
this project. Parameters of our CWBA model for which the SPD metric performs
better than the DSD metric could provide insights into characteristics of
scale-free graphs that help prediction methods using the DSD metric to work
well. Another question to be studied could be how centrality and different
measures on graphs are similar the the DSD. Measures on graphs are different
from metrics in that measures describe attributes of a single graph vertex
rather than the relation between two vertices. An example of a graph measure
that could be studied is betweenness centrality.
The effectiveness of the DSD in detecting manifold-like structures of
graphs is also an interesting question. Several simulation models could be
constructed to study this question. Also, the effect of the DSD metric in
detecting the dispersion of rumors or the spread of information could also
be studied. The way the DSD metric is defined by random walks seems to give
a hint for what kinds of real-world data that it should be experimented with.
The DSD metric could be compared to heat dispersion and different kinds
of dispersions in nature, such as the dispersion of particles or biological
dispersion of pollen and seeds. Such studies could provide a more accurate
understanding of how natural biological networks interact with each other.